% Copyright 2016 by Wang Kunzhen <wangkunzhen1993@gmail.com>.
%
% This is a latex template adapted from Till Tantau's Beamer template.
% It adds theme customizations for the convenience of users from the
% National University of Singapore. 
% 
% In principle, this file can be redistributed and/or modified under
% the terms of the GNU Public License, version 2.
%
% However, this file is supposed to be a template to be modified
% for your own needs. For this reason, if you use this file as a
% template and not specifically distribute it as part of a another
% package/program, I grant the extra permission to freely copy and
% modify this file as you see fit and even to delete this copyright
% notice. 

\documentclass[xcolor=dvipsnames,professionalfonts,ignorenonframetext,aspectratio=169,10pt]{beamer}

\usetheme{AnnArbor}

\definecolor{hkbu-red}{RGB}{239,124,0}
% \definecolor{hkbu-orange}{RGB}{217, 178, 125}
\definecolor{hkbu-orange}{RGB}{211, 175, 126}
\definecolor{hkbu-white}{RGB}{255,255,255}
% \definecolor{hkbu-blue}{RGB}{0, 72, 157}
\definecolor{hkbu-blue}{RGB}{0, 82, 155}
\definecolor{hkbu-black}{RGB}{0,0,0}
\definecolor{beamer@gray}{RGB}{210,210,210}
\definecolor{beamer@gray2}{RGB}{244,244,244}

% Uncomment this section if you want the title background for each slide to be gradient like decaying from hkbu-orange to hkbu-white.
% \useoutertheme{shadow}
% \usepackage{tikz}
% \usetikzlibrary{shadings}
% \colorlet{titleleft}{hkbu-orange}
% \colorlet{titleright}{hkbu-orange!45!hkbu-white}
% \makeatletter
% \pgfdeclarehorizontalshading[titleleft,titleright]{beamer@frametitleshade}{\paperheight}{%
%   color(0pt)=(titleleft);
%   color(\paperwidth)=(titleright)}
% \makeatother
% End of gradient slide title effect.

\setbeamercolor{section in head/foot}{bg=hkbu-blue, fg=hkbu-white}
\setbeamercolor{subsection in head/foot}{bg=hkbu-orange, fg=hkbu-white}
% \setbeamercolor{frametitle}{bg=hkbu-orange, fg=hkbu-black}
\setbeamercolor{frametitle}{bg=hkbu-orange, fg=hkbu-white}
\setbeamercolor{title}{bg=hkbu-orange, fg=hkbu-white}
\setbeamercolor{alerted text}{fg=hkbu-red}
\setbeamercolor{block title}{fg=hkbu-blue}
\setbeamercolor{block body}{fg=hkbu-black}

% Setup blocks
\setbeamercolor{block title}{fg = white, bg = hkbu-blue}
\setbeamercolor{block body}{fg=black, bg=beamer@gray2}

\setbeamercolor{block title alerted}{fg = white, bg = hkbu-orange}
\setbeamercolor{block body alerted}{fg=black, bg=beamer@gray2}

\setbeamercolor{block title example}{fg = black, bg = beamer@gray}
\setbeamercolor{block body example}{fg=black,bg=beamer@gray2}

\setbeamercolor{item projected}{fg=white,bg=hkbu-blue}


\setbeamertemplate{theorems}[numbered]
\setbeamertemplate{propositions}[numbered]

\setbeamertemplate{bibliography item}{\insertbiblabel}

% \setbeamertemplate{itemize items}[square]
\setbeamertemplate{itemize item}{\color{hkbu-blue}$\bullet$}

\setbeamertemplate{headline}{}
% \setbeamertemplate{footline}[frame number]
\makeatletter
\setbeamertemplate{footline}
{
  \leavevmode%
  \hbox{%
  \begin{beamercolorbox}[wd=.333333\paperwidth,ht=2.25ex,dp=1ex,center]{section in head/foot}%
    \usebeamerfont{author in head/foot}\insertshortauthor~~\beamer@ifempty{\insertshortinstitute}{}{(\insertshortinstitute)}
  \end{beamercolorbox}%
  \begin{beamercolorbox}[wd=.333333\paperwidth,ht=2.25ex,dp=1ex,center]{subsection in head/foot}%
    \usebeamerfont{title in head/foot}\insertshorttitle
  \end{beamercolorbox}%
  \begin{beamercolorbox}[wd=.333333\paperwidth,ht=2.25ex,dp=1ex,right]{section in head/foot}%
    \usebeamerfont{date in head/foot}\insertsection\hfill%
    \insertframenumber{} / \inserttotalframenumber\hspace*{2ex} 
  \end{beamercolorbox}}%
  \vskip0pt%
}
\makeatother


\usepackage{eso-pic}
\usepackage{svg}

\beamertemplatenavigationsymbolsempty

% \newcommand\AtPagemyUpperLeft[1]{\AtPageLowerLeft{%
% \put(\LenToUnit{0.71\paperwidth},\LenToUnit{0.91\paperheight}){#1}}}
% \AddToShipoutPictureFG{
%   % \AtPagemyUpperLeft{{\includegraphics[width=3.5cm,keepaspectratio]{logo.png}}}
%   \AtPagemyUpperLeft{{\includegraphics[width=3.5cm,keepaspectratio]{Daco_5429558 - Copy.png}}}
% }%


\setbeamertemplate{title page}[default][colsep=-4bp,rounded=true, shadow=true]

\title[A Beamer Template for HKBU]{A Beamer Template for \\
Hong Kong Baptist University}

% \subtitle{Sub-title}

\author[First Last]
{First Last}

% \institute[HKBU]{\includegraphics[width=6cm]{hkbu-comp.png}}
\institute[HKBU]{
\includegraphics[height=0.7cm]{Daco_5429558.png}
}

\date{\today}

\begin{document}

\begin{frame}
  \titlepage
\end{frame}

\begin{frame}{Outline}
  \tableofcontents
\end{frame}

\section{First Main Section}

\subsection{First Subsection}
\begin{frame}{First Slide Title}{Optional Subtitle}
  \begin{itemize}
  \item {
    My first point.
  }
  \item {
    My second point.
  }
  \end{itemize}
\end{frame}

\subsection{Second Subsection}
% You can reveal the parts of a slide one at a time
% with the \pause command:
\begin{frame}{Second Slide Title}
  \begin{itemize}
  \item {
    First item.
    \pause % The slide will pause after showing the first item
  }
  % You can also specify when the content should appear
  % by using <n->:
  \item<3-> {
    Third item.
  }
  % or you can use the \uncover command to reveal general
  % content (not just \items):
  \item<5-> {
    Fifth item. \uncover<6->{Extra text in the fifth item.}
  }
  \end{itemize}
\end{frame}

\section{Second Main Section}

\subsection{Second Subsection}
\begin{frame}{Main Theorem}
\begin{theorem}
Theorem Statements. Example for citation \cite{Author1990}.
\end{theorem}

\begin{proof}
Proof of the theorem goes here.
\end{proof}
\end{frame}

% Placing a * after \section means it will not show in the
% outline or table of contents.
\section*{Summary}

\begin{frame}{Summary}
  \begin{itemize}
  \item
    The \alert{first main message} of your talk in one or two lines.
  \item
    The \alert{second main message} of your talk in one or two lines.
  \item
    Perhaps a \alert{third message}, but not more than that.
  \end{itemize}
  
  \begin{itemize}
  \item
    Outlook
    \begin{itemize}
    \item
      Something you haven't solved.
    \item
      Something else you haven't solved.
    \end{itemize}
  \end{itemize}
\end{frame}

% Bibliography section. Use \bibitem to add more bibliography items.
\section*{Bibliography}
\begin{frame}{Bibliography}
  \begin{thebibliography}{10}

  \bibitem{Author1990}
    A.~Author.
    \newblock {\em Handbook of Everything}.
    \newblock Some Press, 1990.

  \bibitem{Someone2000}
    S.~Someone.
    \newblock On this and that.
    \newblock {\em Journal of This and That}, 2(1):50--100,
    2000.

  \end{thebibliography}
\end{frame}

\end{document}